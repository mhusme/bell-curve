%%%%%%%%%%%%%%%%%%%%%%%%%%%%%%%%%%%%%%%%%%%%%%%%%%%%%%%%%%%%%%%
%
% Welcome to Overleaf --- just edit your LaTeX on the left,
% and we'll compile it for you on the right. If you open the
% 'Share' menu, you can invite other users to edit at the same
% time. See www.overleaf.com/learn for more info. Enjoy!
%
%%%%%%%%%%%%%%%%%%%%%%%%%%%%%%%%%%%%%%%%%%%%%%%%%%%%%%%%%%%%%%%
\documentclass[border=10pt]{standalone}
%%%<
\usepackage{verbatim}
%%%>
\usepackage{pgfplots}
\pgfplotsset{width=7cm,compat=1.8}
\begin{comment}
:Title: Gaussian Function aka Bell curve
:Tags: 2D;Functions;PGFmath;Mathematics
:Author: Jake
:Slug: bell-curve

You can use pgfplots to plot your own functions. There's no standard macro
for it, but the function isn't too complicated and can be added as a
pgfmath function.

This code was written by Jake on TeX.SE.
\end{comment}
\pgfmathdeclarefunction{gauss}{2}{%
  \pgfmathparse{1/(#2*sqrt(2*pi))*exp(-((x-#1)^2)/(2*#2^2))}%
}
\begin{document}
\begin{tikzpicture}
\begin{axis}[every axis plot post/.append style={
  mark=none,domain=-2:3,samples=50,smooth},
    % All plots: from -2:2, 50 samples, smooth, no marks
  axis x line*=bottom, % no box around the plot, only x and y axis
  axis y line*=left, % the * suppresses the arrow tips
  ylabel={Probability},
  xlabel={Message-box access},
  yticklabels={,,},
  xticklabels={,,},
  enlargelimits=upper] % extend the axes a bit to the right and top
  \addplot[draw=black, fill=black, fill opacity=0.2, draw opacity =0.2] {gauss(0.5,0.70)};
  \addlegendentry{$D1$}
  
  \addplot[draw=black, fill=black, fill opacity=0.7, draw opacity =0.7] {gauss(0.7,0.70)};
  \addlegendentry{$D2$}
\end{axis}
\end{tikzpicture}
\end{document}